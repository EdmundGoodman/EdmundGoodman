\documentclass{article}

\usepackage[utf8]{inputenc}
\usepackage{amssymb}
\usepackage{enumitem}
\usepackage{geometry}
 \geometry{
 a4paper,
 total={170mm,257mm},
 left=20mm,
 top=20mm,
}

\usepackage[utf8]{inputenc}
\usepackage{amssymb}
\usepackage{enumitem}





\begin{document}

\vspace*{\stretch{0.5}}
\begin{center}
    \Huge\textbf{Personal Statement}\\
    \huge\textit{Edmund Goodman}\\
    \huge\textit{October, 2020}
\end{center}
\vspace*{\stretch{0.5}}
%Vspace seems to shift dependent on the size available on the page

I am currently studying the MEng Computer Science course at Warwick, and enjoy playing music, climbing, and (unsurprisingly) computer science. I am both interested and have experience in cyber security, networking, and robotics, and am looking to further my knowledge in those subjects.

I attended the Perse school with an academic scholarship, studying A-levels in Maths, Further Maths, Physics and Computer Science, acheiving a D2 (eq. A*), D3 (eq. A*/A), D1 (eq. high A*), and an A* respectively. Before that, I acheived GCSEs in Mathematics (9), Physics (9), Biology (9), English Language (9), English Literature (8), Chemistry (A*), Computer Science (A*), Design \& Technology (A*), French (A*), History (A*), Music (A*), a HPQ on computer security protocols (A*), and a Free Standing Maths Qualification (A - highest grade)

I am fluent in Python, Java, Bash, VBA, Javascript with jQuery, BASIC, PHP, HTML, CSS, SQL, and have some experience with C, C++, and Haskell.

I have completed various programming projects (source code at https://github.com/EdmundGoodman), such as:
an implementation of Conway’s game of life, Markov chains to randomly generate human readable text, a first principles neural network, a minimax algorithm with alpha-beta pruning, and a genetic algorithm all to play abstract games, and programs to optimally play board games, such as hangman, boggle, and probe. Alongside these programming projects, I also enjoy practical projects, for example: I built a six-axis robot arm for my GCSE technology coursework, and I led the software aspect of a school "Engineering Education Scheme" entry, where we built an autonomous tennis ball collecting robot, identifying balls from webcam images, and interfacing with motors collect them, and received gold awards from both CREST and Industrial Cadets. As a result of this I have experience in hardware control and computer vision, and learned the importance of systematic, thorough testing, and the careful consideration of edge cases. This set of skills is complemented by my interest in cyber security. I have competed in a national cyber security competition (https://joincyberdiscovery.com), and have been selected for the final stage, a residential camp for high-achieving competitors. In the first year, of 23,000 competitors, I was one of 170 selected. Last year, of 30,000, I was one of 180 selected to take a week-long course - in my case “SANS SEC504 - incident management and hacking tools”, and passed its associated GIAC examination, in the youngest cohort ever to do so. This year, of 70,000, I was one of 240 selected to take a similar course - "SANS FOR500 - windows forensics analysis", and am yet to take the exam.

I was awarded 1st place in the Caius college engineering essay contest, with an entry titled "Powering the UK's electricity supply - distributed systems to enhance the viability of renewable power". I attained a gold award for the AS physics challenge, and a commendation for the A2 physics Olympiad, after being put forward to take it a year early. I have also received a distinction in the "Bebras Computational Thinking Challenge Senior Category". Participating in these competitions has improved my logical reasoning, and ability to work under pressure. I have passed Trinity Grade 8 Trumpet, and attained a merit in ABRSM Grade 7 Singing, and played in a large number of school music groups, including a brass ensemble that was a finalist in a national chamber music competition, and a chamber choir which allowed me to sing a solo in John's College Chapel Cambridge. I was head chorister of Jesus College Choir, recording three CDs and going on three international tours, and was a Sergeant in my school CCF section. Both roles have given me strong leadership abilities, which I developed by completing the highly competitive RAF Air Cadet Leadership Course, a week long course at RAF Cranwell which only around 240 cadets from around the country graduate from each year, where I learned how to plan an exercise, communicate effectively, and command a team of nine other cadets.


I took a three month internship at "HUBER+SUHNER Polatis", where I updated a testing harness for fibre optic switches from python2 to python3, refactored the VBA backend of an excel spreadsheet used for corporate planning, created visual representations for the aforementioned planning data, and wrote a Java implementation of NETCONF call home (https://tools.ietf.org/html/rfc8071) for the open source ONOS project, which is in the process of being upstreamed into the main codebase. Before that, I took a two week work experience placement at “Argon Design”, where I refactored and redesigned an information display board, converting a monolithic file into a series of microservices running with a python LAMP server, in order to serve the updated UI.



\end{document}
